\chapter{Holomorphic square roots and logarithms}
\label{ch:complex_log}
In this chapter we'll make sense of a holomorphic square root and logarithm.
The main results are \Cref{thm:nth_root}, \Cref{thm:holomorphic_log},
\Cref{cor:nonvanishing}, and \Cref{cor:principal}.
If you like, you can read just these four results, and skip the discussion of how they came to be.

Let $f : U \to \CC$ be a holomorphic function.
A \vocab{holomorphic $n$th root} of $f$ is a function $g : U \to \CC$
such that $f(z) = g(z)^n$ for all $z \in U$.
A \vocab{logarithm} of $f$ is a function $g : U \to \CC$
such that $f(z) = e^{g(z)}$ for all $z \in U$.
The main question we'll try to figure out is: when do these exist?
In particular, what if $f = \id$?

\section{Motivation: square root of a complex number}
To start us off, can we define $\sqrt z$ for any complex number $z$?

The first obvious problem that comes up is that for any $z$, there are
\emph{two} numbers $w$ such that $w^2 = z$.
How can we pick one to use?
For our ordinary square root function, we had a notion of ``positive'',
and so we simply took the positive root.

Let's expand on this: given
$ z = r \left( \cos\theta + i \sin\theta \right) $
(here $r \ge 0$) we should take the root to be
\[ w = \sqrt{r} \left( \cos \alpha + i \sin \alpha \right). \]
such that $2\alpha \equiv \theta \pmod{2\pi}$;
there are two choices for $\alpha \pmod{2\pi}$, differing by $\pi$.

For complex numbers, we don't have an obvious way to pick $\alpha$.
Nonetheless, perhaps we can also get away with an arbitrary distinction:
let's see what happens if we just choose the $\alpha$ with $-\half\pi < \alpha \le \half\pi$.

Pictured below are some points (in red)
and their images (in blue) under this ``upper-half'' square root.
The condition on $\alpha$ means we are forcing the blue points to lie on the right-half plane.

\begin{center}
	%% First Diagram
	\begin{asy}
		size(7cm);
		cplane();
		draw(origin--(4.2*dir(180)), grey+2);
		pair O = (-2.4, 1.1);
		pair T;
		pair[] A = new pair[8];
		real r = 1.44;
		real[] dirs = {45,45,45,135,175,255,-90,-75};
		for (int i=0; i<8; ++i) {
			A[i] = O + r*dir(i*45);
			dot("$z_{" + (string) i + "}$", A[i], dir(dirs[i]), red);
		}
		draw(A[0]..A[1]..A[2]..A[3]..A[4]..A[5]..A[6]..A[7]..cycle, red);

		real[] dirs = {-45, -15, 15, 45, 75, 150, 210, 230};
		for (int i=0; i<8; ++i) {
			A[i] = sqrt(abs(A[i]))*dir(degrees(A[i])/2);
			if (degrees(A[i]) > 90) { A[i] *= -1; }
			dot("$w_{" + (string) i + "}$", A[i], dir(dirs[i]), blue);
		}
		draw(A[7]..A[0]..A[1]..A[2]..A[3]..A[4]..A[5], blue);
	\end{asy}
\end{center}


Here, $w_i^2 = z_i$ for each $i$, and we are constraining the $w_i$
to lie in the right half of the complex plane.
We see there is an obvious issue: there is a big discontinuity near
the points $w_5$ and $w_7$!
The nearby point $w_6$ has been mapped very far away.
This discontinuity occurs since the points on the negative real axis are
at the ``boundary''.
For example, given $-4$, we send it to $-2i$, but we have hit the boundary:
in our interval $-\half\pi \le \alpha < \half\pi$, we are at the very left edge.


The negative real axis that we must not touch
is what we will later call a \emph{branch cut},
but for now I call it a \textbf{ray of death}.
It is a warning to the red points: if you cross this line, you will die!
However, if we move the red circle just a little upwards
(so that it misses the negative real axis) this issue is avoided entirely,
and we get what seems to be a ``nice'' square root.

\begin{center}
	%% Second Diagram
	\begin{asy}
		size(7cm);
		cplane();
		draw(origin--(4.2*dir(180)), grey+2);
		pair O = (-2.8, 1.9);
		pair T;
		pair[] A = new pair[8];
		real r = 1.44;
		real[] dirs = {45,45,45,135,175,225,-45,-45};
		for (int i=0; i<8; ++i) {
			A[i] = O + r*dir(i*45);
			dot("$z_{" + (string) i + "}$", A[i], dir(dirs[i]), red);
		}
		draw(A[0]..A[1]..A[2]..A[3]..A[4]..A[5]..A[6]..A[7]..cycle, red);

		real[] dirs = {-45, -15, 15, 75, 105, 150, 210, 260};
		for (int i=0; i<8; ++i) {
			A[i] = sqrt(abs(A[i]))*dir(degrees(A[i])/2);
			if (degrees(A[i]) > 90) { A[i] *= -1; }
			dot("$w_{" + (string) i + "}$", A[i], dir(dirs[i]), blue);
		}
		draw(A[0]..A[1]..A[2]..A[3]..A[4]..A[5]..A[6]..A[7]..cycle, blue);
	\end{asy}
\end{center}

In fact, the ray of death is fairly arbitrary:
it is the set of ``boundary issues''
that arose when we picked $-\half\pi < \alpha \le \half\pi$.
Suppose we instead insisted on the interval $0 \le \alpha < \pi$;
then the ray of death would be the \emph{positive} real axis instead.
The earlier circle we had now works just fine.

\begin{center}
	%% Third Diagram
	\begin{asy}
		size(7cm);
		cplane();
		draw(origin--(4.2*dir(0)), grey+2);
		pair O = (-2.4, 1.1);
		pair T;
		pair[] A = new pair[8];
		real r = 1.44;
		real[] dirs = {-5,45,45,135,175,255,-90,-75};
		for (int i=0; i<8; ++i) {
			A[i] = O + r*dir(i*45);
			dot("$z_{" + (string) i + "}$", A[i], dir(dirs[i]), red);
		}
		draw(A[0]..A[1]..A[2]..A[3]..A[4]..A[5]..A[6]..A[7]..cycle, red);

		real[] dirs = {-45, -15, 15, 45, 75, 150, 180, 230};
		for (int i=0; i<8; ++i) {
			A[i] = sqrt(abs(A[i]))*dir(degrees(A[i])/2);
			dot("$w_{" + (string) i + "}$", A[i], dir(dirs[i]), blue);
		}
		draw(A[0]..A[1]..A[2]..A[3]..A[4]..A[5]..A[6]..A[7]..cycle, blue);
	\end{asy}
\end{center}

What we see is that picking a particular $\alpha$-interval
leads to a different set of edge cases, and hence a different ray of death.
The only thing these rays have in common is their starting point of zero.
In other words, given a red circle and a restriction of $\alpha$,
I can make a nice ``square rooted'' blue circle as long as the
ray of death misses it.

So, what exactly is going on?

\section{Square roots of holomorphic functions}
To get a picture of what's happening, we would like to consider a more
general problem: let $f: U \to \CC$ be holomorphic.
Then we want to decide whether there is a $g : U \to \CC$
such that \[ f(z) = g(z)^2. \]
Our previous discussion with $f = \id$
tells us we cannot hope to achieve this for $U = \CC$;
there is a ``half-ray'' which causes problems.
However, there are certainly functions $f : \CC \to \CC$ such that a $g$ exists.
As a simplest example, $f(z) = z^2$ should definitely have a square root!

Now let's see if we can fudge together a square root.
Earlier, what we did was try to specify a rule to force one of the two choices at each point.
This is unnecessarily strict.
Perhaps we can do something like:
start at a point in $z_0 \in U$, pick a square root $w_0$ of $f(z_0)$,
and then try to ``fudge'' from there the square roots of the other points.
What do I mean by fudge?
Well, suppose $z_1$ is a point very close to $z_0$,
and we want to pick a square root $w_1$ of $f(z_1)$.
While there are two choices, we also would expect $w_0$ to be close to $w_1$.
Unless we are highly unlucky, this should tell us which choice of $w_1$ to pick.
(Stupid concrete example: if I have taken the square root $-4.12i$ of $-17$
and then ask you to continue this square root to $-16$,
which sign should you pick for $\pm 4i$?)

There are two possible ways we could get unlucky in the scheme above:
first, if $w_0 = 0$, then we're sunk.
But even if we avoid that, we have to worry that
if we run a full loop in the complex plane,
we might end up in a different place from where we started.
For concreteness, consider the following situation, again with $f = \id$:

\begin{center}
	%% Fourth Diagram
	\begin{asy}
		size(7cm);
		cplane();
		pair O = origin;
		pair T;
		pair[] A = new pair[8];
		real r = 3.24;
		real[] dirs = {45,45,45,135,135,255,-45,-45};
		for (int i=0; i<8; ++i) {
			A[i] = O + r*dir(i*45);
			dot("$z_{" + (string) i + "}$", A[i], dir(dirs[i]), red);
		}
		draw(A[0]..A[1]..A[2]..A[3]..A[4]..A[5]..A[6]..A[7]..cycle, red);

		real[] dirs = {-65, -15, 15, 45, 45, 135, 143, 135};
		for (int i=0; i<8; ++i) {
			A[i] = sqrt(abs(A[i]))*dir(degrees(A[i])/2);
			dot("$w_{" + (string) i + "}$", A[i], dir(dirs[i]), blue);
		}
		draw(A[0]..A[1]..A[2]..A[3]..A[4]..A[5]..A[6]..A[7]..(-A[0]), blue);
	\end{asy}
\end{center}

We started at the point $z_0$, with one of its square roots as $w_0$.
We then wound a full red circle around the origin, only to find that at the end of it,
the blue arc is at a different place where it started!

The interval construction from earlier doesn't work either:
no matter how we pick the interval for $\alpha$,
any ray of death must hit our red circle.
The problem somehow lies with the fact that we have enclosed the 
very special point $0$.

Nevertheless, we know that if we take $f(z) = z^2$, then we don't run into any problems
with our ``make it up as you go'' procedure.
So, what exactly is going on?

\section{Covering projections}
By now, if you have read the part on algebraic topology,
this should all seem quite familiar.
The ``fudging'' procedure exactly describes the idea of a lifting.

More precisely, recall that there is a covering projection
\[ (-)^2 : \CC \setminus \{0\}
	\to \CC \setminus \{0\}. \]
Let $V = \left\{ z \in U \mid f(z) \neq 0 \right\}$.
For $z \in U \setminus V$, we already have the square root $g(z) = \sqrt{f(z)} = \sqrt 0 = 0$.
So the burden is completing $g : V \to \CC$.

Then essentially, what we are trying to do is construct a lifting $g$ in the diagram
\begin{diagram}
	&&& E = \CC \setminus \{0\} \\
	&& \ruTo(3,2)^{g} & \dTo_{p = (-)^2} \\
	V & \rTo_f && B = \CC \setminus \{0\}.
\end{diagram}

Our map $p$ can be described as ``winding around twice''.
Our \Cref{thm:lifting} now tells us that this lifting exists if and only if
\[ f_\ast\im(\pi_1(V)) \subseteq p_\ast\im(\pi_1(E)) \]
is a subset of the image of $\pi_1(E)$ by $p$.
Since $B$ and $E$ are both punctured planes, we can identify them with $S^1$.
\begin{ques}
	Show that the image under $p$ is
	exactly $2\ZZ$ once we identify $\pi_1(B) = \ZZ$.
\end{ques}

That means that for any loop $\gamma$ in $V$, we need $f \circ \gamma$ to have an
\emph{even} winding number around $0 \in B$.
This amounts to
\[
	\frac{1}{2\pi} \oint_\gamma \frac{f'}{f} \; dz \in 2\ZZ
\]
since $f$ has no poles.

Replacing $2$ with $n$ and carrying over the discussion gives the first main result.
\begin{theorem}
	[Existence of holomorphic $n$th roots]
	\label{thm:nth_root}
	Let $f : U \to \CC$ be holomorphic.  
	Then $f$ has a holomorphic $n$th root if and only if 
	\[ \frac{1}{2\pi i}\oint_\gamma \frac{f'}{f} \; dz \in n\ZZ \]
	for every contour $\gamma$ in $U$.
\end{theorem}


\section{Complex logarithms}
The multivalued nature of the complex logarithm comes from the fact that
\[ \exp(z+2\pi i) = \exp(z). \]
So if $e^w = z$, then any complex number $w + 2\pi i k$ is also a solution.

We can handle this in the same way as before:
it amounts to a lifting of the following diagram.
\begin{diagram}
	&&& E = \CC \\
	&& \ruTo(3,2)^{g} & \dTo_{p = \exp} \\
	U & \rTo_f && B = \CC \setminus \{0\}
\end{diagram}
There is no longer a need to work with a separate $V$ since:
\begin{ques}
	Show that if $f$ has any zeros then $g$ can't possibly exist.
\end{ques}
In fact, the map $\exp : \CC \to \CC\setminus\{0\}$ is a universal cover,
since $\CC$ is simply connected.
Thus, $p\im(\pi_1(\CC))$ is \emph{trivial}.
So in addition to being zero-free, $f$ cannot have any winding number around $0 \in B$ at all.
In other words:
\begin{theorem}
	[Existence of logarithms]
	\label{thm:holomorphic_log}
	Let $f : U \to \CC$ be holomorphic.
	Then $f$ has a logarithm if and only if
	\[ \frac{1}{2\pi i}\oint_\gamma \frac{f'}{f} \; dz  = 0 \]
	for every contour $\gamma$ in $U$.
\end{theorem}

\section{Some special cases}
The most common special case is
\begin{corollary}
	[Nonvanishing functions from simply connected domains]
	\label{cor:nonvanishing}
	Let $f : \Omega \to \CC$ be continuous, where $\Omega$ is simply connected.
	If $f(z) \neq 0$ for every $z \in \Omega$,
	then $f$ has both a logarithm and holomorphic $n$th root.
\end{corollary}

Finally, let's return to the question of $f = \id$ from the very beginning.
What's the best domain $U$ such that \[ \sqrt{-} : U \to \CC \]
is well-defined?  Clearly $U = \CC$ cannot be made to work,
but we can do almost as well.
For note that the only zero of $f = \id$ is at the origin.
Thus if we want to make a logarithm exist,
all we have to do is make an incision in the complex plane
that renders it impossible to make a loop around the origin.
The usual choice is to delete negative half of the real axis, our very first ray of death;
we call this a \vocab{branch cut}, with \vocab{branch point} at $0 \in \CC$
(the point which we cannot circle around).
This gives
\begin{theorem}[Branch cut functions]
	\label{cor:principal}
	There exist holomorphic functions
	\begin{align*}
		\log &: \CC \setminus (-\infty, 0] \to \CC \\ % chktex 9
		\sqrt[n]{-} &: \CC \setminus (-\infty, 0] \to \CC % chktex 9
	\end{align*}
	satisfying the obvious properties.
\end{theorem}
There are many possible choices of such functions ($n$ choices for the $n$th root
and infinitely many for $\log$);
a choice of such a function is called a \vocab{branch}.
So this is what is meant by a ``branch'' of a logarithm.

The \vocab{principal branch} is the ``canonical'' branch, analogous
to the way we arbitrarily pick the positive branch
to define $\sqrt{-} : \RR_{\ge 0} \to \RR_{\ge 0}$.
For $\log$, we take the $w$ such that $e^w = z$
and the imaginary part of $w$ lies in $(-\pi, \pi]$ % chktex 9
(since we can shift by integer multiples of $2\pi i$).
Often, authors will write $\opname{Log} z$ to emphasize this choice.


\section\problemhead
\begin{problem}
	Show that a holomorphic function $f : U \to \CC$
	has a holomorphic logarithm if and only if it
	has a holomorphic $n$th root for every integer $n$.
\end{problem}

\begin{problem}
	Show that the function $f : U \to \CC$
	by $z \mapsto z(z-1)$ has a holomorphic square root,
	where $U$ is the entire complex plane minus the closed interval $[0,1]$.
\end{problem}
